%-------------------------
% Entry-level Resume in LaTeX
% Version - v1.3
% Last Edits - October 5, 2021
% Author : Jayesh Sanwal
% Reach out to me on LinkedIn(/in/jsanwal), with any suggestions, ideas, issues, etc.
%------------------------

%%-------------------------------------
% Notes to the User here -
% 1) Change " Author+an = {N=highlight}" in "Publications.bib", where N is the number at which your name appears
% 2) Use "\vspace" wisely, that would change spacing, and is currently being used as a hacky fix
% 3) Do not delete the fonts in the left side
% 4) Use the "Rich Text" feature on Overleaf - On the top panel next to source. Makes it much easier for starters on LaTeX to use this template
% 5) Do NOT use periods at the end of bullet points, the sample (ipsum) text might have it
% 6) Use "maxbibnames" on this file to change the maximum number of authors on the paper (Credits: Dr. Natasha Krell) - Default is 3, but change line 92 to add authors

%%-------------------------------------
% Changes from last version (v1.3, October 5, 2021) -
% 1) Summary statement removed, replaced with 4 keywords on top and Impact statement

% Changes from last version (v1.2, August 23, 2021) -
% 1) Changes in the Technical Skills section - renamed to just Skills
% 2) Changed the font size for the name
% 3) Education to the top
%%-------------------------------------


%%-------------------------------------
% Changes to be made in the next version -
% 1) Shift to Class file, make changes here
% 2) Use a fonts folder
% 3) Incorporate Leadership & volunteering together
% 4) Remove \vspace based 'hacky' fixes
%%-------------------------------------


\documentclass[a4,10pt]{article}

%%%%%%% --------------------------------------------------------------------------------------
%%%%%%%  STARTING HERE, DO NOT TOUCH ANYTHING 
%%%%%%% --------------------------------------------------------------------------------------

\usepackage{latexsym}
\usepackage{fontawesome5}
\usepackage[empty]{fullpage}
\usepackage{titlesec}
 \usepackage{marvosym}
\usepackage[usenames,dvipsnames]{color}
\usepackage{verbatim}
\usepackage[hidelinks]{hyperref}
\usepackage{fancyhdr}
\usepackage{multicol}
\usepackage{hyperref}
\usepackage{csquotes}
\usepackage{tabularx}
\hypersetup{colorlinks=true,urlcolor=blue}
\usepackage[11pt]{moresize}
\usepackage{setspace}
\usepackage{fontspec}
\usepackage[inline]{enumitem}
\usepackage{array}
\newcolumntype{P}[1]{>{\centering\arraybackslash}p{#1}}
\usepackage{anyfontsize}
\usepackage{xcolor}
\usepackage{graphicx}
\usepackage{grffile}

%%%% Set Margins
\usepackage[margin=1cm, top=1cm]{geometry}

%%%% Set Fonts
\setmainfont[
BoldFont=SourceSansPro-Semibold.otf, %SourceSansPro-Bold.otf
ItalicFont=SourceSansPro-RegularIt.otf
]{SourceSansPro-Regular.otf}
\setsansfont{SourceSansPro-Semibold.otf}

%%%% Set Page
\pagestyle{fancy}
\fancyhf{} 
\fancyfoot{}
\renewcommand{\headrulewidth}{0pt}
\renewcommand{\footrulewidth}{0pt}

%%%% Set URL Style
\urlstyle{same}

%%%% Set Indentation
\raggedbottom
\raggedright
\setlength{\tabcolsep}{0in}

%%%% Set Secondary Color
\definecolor{UI_blue}{RGB}{32, 64, 151}
\definecolor{Text}{RGB}{34, 34, 34}
% link color
\definecolor{LINK_blue}{RGB}{29, 58, 136}
\definecolor{WHITE}{RGB}{255,255,255}

%%%% Define New Commands
\usepackage[style=nature, maxbibnames=3]{biblatex}
\addbibresource{Publications.bib}

%%%% Bold Name in Publications
\renewcommand*{\mkbibnamegiven}[1]{%
\ifitemannotation{highlight}
{\textbf{#1}}
{#1}}

\renewcommand*{\mkbibnamefamily}[1]{%
\ifitemannotation{highlight}
{\textbf{#1}}
{#1}}

%%%% Set Sections formatting
\titleformat{\section}{
\color{Text} \scshape \raggedright \Large 
}{}{0em}{}[\vspace{-10pt} \hrulefill \vspace{-6pt}]

% \titlespacing*{\section}{0pt}{6pt plus 1pt minus 1pt}{*}



%%%% Set Subtext Formatting
\newcommand{\subtext}[1]{
#1\par\vspace{-0.2cm}}

% \newcommand{\subtextit}[1]{\vspace{0.15cm}
% \textit{ #1 \vspace{-0.2cm}} }

%%%% Set Item Spacing
\setlist[itemize]{align=parleft,left=0pt..1em}

%%%% New Itemize "Zitemize" Formatting - tighter spacing than itemize
\newenvironment{zitemize}{
\begin{itemize}\itemsep0pt \parskip0pt \parsep1pt}
{\end{itemize}\vspace{-0.5cm}}

%%%% Define Skills Bold Formatting
\newcommand{\hskills}[1]{
\textbf{\bfseries #1} }

%%%% Set Subsection formatting
\titleformat{\subsection}{\vspace{-0.1cm} 
\bfseries \fontsize{13pt}{2cm}}{}{0em}{}[\vspace{-0.2cm}]

%%%%%%% --------------------------------------------------------------------------------------
%%%%%%% --------------------------------------------------------------------------------------
%%%%%%%  END OF "DO NOT TOUCH" REGION
%%%%%%% --------------------------------------------------------------------------------------
%%%%%%% --------------------------------------------------------------------------------------




\begin{document}
%%%%%%% --------------------------------------------------------------------------------------
%%%%%%%  HEADER
%%%%%%% --------------------------------------------------------------------------------------
% #204097


%%%%%%% --------------------------------------------------------------------------------------
%%%%%%%  HEADER
%%%%%%% --------------------------------------------------------------------------------------

\begin{flushleft}
    {\Huge \bfseries Shyren More} \\
    
    \vspace{2pt}
    
    \faLinkedin\, \href{https://linkedin.com/in/shyrenmore/}{linkedin.com/in/shyrenmore} \hspace{1em}
    \faGithub\, \href{https://github.com/ShyrenMore}{github.com/ShyrenMore} \hspace{1em}
    \faEnvelope\, \href{mailto:shyren.more30@gmail.com}{shyren.more30@gmail.com} \hspace{1em}
    \faPhone\, \href{tel:+918169622410}{+91 81696 22410}

    \vspace{4pt}

    Data structure and algorithms profiles:
    \href{https://leetcode.com/shyren_more/}{Leetcode} | 
    \href{https://takeuforward.org/plus/profile/shyren_more}{TUF} | 
    \href{https://www.geeksforgeeks.org/user/shyrenmore30/}{GeeksForGeeks}
\end{flushleft}



%%%%%%% --------------------------------------------------------------------------------------
%%%%%%%  WORK EXPERIENCE
%%%%%%% --------------------------------------------------------------------------------------
\vspace{-18pt}
\section{Work Experience}

\subsection*{Barclays Bank PLC \hfill {\normalsize\normalfont Pune, Maharashtra, India}}

\textbf{Software Engineer 2 (BA4)} \hfill \textit{03/2025 - Present} \\
\vspace{-5pt}
\begin{zitemize}
    \item Co-created a \textbf{real-time payments authorization module for the \href{https://www.amazon.co.uk/dp/B0BH98211K}{Amazon Barclaycard}}, achieving 70 transactions per second with a 60 ms P95 response time, leveraging Java, Spring Boot, and Amazon Web Services (AWS) for scalability and reliability.
    \item Engineered an optimized \textbf{FCA-compliant Persistent Debt} logic on the next-gen credit card platform by restricting evaluations to eligible accounts, resulting in 85\% memory and processing savings over the legacy system utilizing AWS Lambda functions and ECS.
    \item Architected and developed an account onboarding feature using SAGA orchestration and the transactional outbox pattern, enabling seamless integration across Treatments, Payments, and Transaction Processing domains.
\end{zitemize}

\vspace{8pt} 

\textbf{Software Developer (BA3)} \hfill \textit{07/2023 - 02/2025} \\
\vspace{-5pt}
\begin{zitemize}
    % \item Implemented a feature for restricting credit card payments based on specific account status in Payments Domain, delivered 100\% of the feature suite by implementing test cases, monitoring, observability, alarms and failure handling along with main feature.
    \item Executed the migration of microservice components to Java 17 and Spring Boot 3, enhancing system performance.
    \item Created alarms, metrics, dashboards, and a logging framework using AWS CloudWatch for effective production monitoring.
    \item Achieved 100\% test coverage for new code using JUnit, Mockito \& adopted BDD techniques to create clear feature scenarios, enhancing collaboration with developers, analysts, and QA teams.
\end{zitemize}


\subsection*{VPSie Inc. \hfill {\normalsize\normalfont New York, United States (Remote)}}

\textbf{Developer Intern} \hfill \textit{06/2022‑ 07/2022} \\
\vspace{-5pt}
\begin{zitemize}
    \item Engineered a secure \textbf{E-KYC system using face recognition} and optical character recognition (OCR) using machine learning techniques for identity card data extraction and user verification through recorded video and extracted identity card photo with a team of 4 developers.
    \item Delivered a 4-step client-side form with a custom drop zone, reactive OTP input, and video capture via MediaRecorder API.
\end{zitemize}


%%%%%%% ----------------------------------- Role 4 ----------------------------------- %%%%%%%
\subsection*{Cyboard School (EdTech Startup) \hfill {\normalsize\normalfont Delhi, India (Remote)}}

\textbf{Web Developer Intern} \hfill \textit{06/2021‑ 09/2021} \\
\vspace{-5pt}
\begin{zitemize}
    \item Designed a responsive web UI with a performance score of 90+ on Google Lighthouse using HTML, CSS, Bootstrap, JS, and packages like AOS, sweet-alert, and Glightbox. 
    \item Optimized homepage performance score on Desktop by 200\% using speed optimization strategies like lazy loading, pre-fetching/pre-loading, and caching that helped improve page ranking on search engines.
\end{zitemize}




%%%%%%% --------------------------------------------------------------------------------------
%%%%%%%  EDUCATION
%%%%%%% --------------------------------------------------------------------------------------
\vspace{-5pt}
\section*{Education}

\textbf{Thadomal Shahani Engineering College} \hfill Mumbai, India \\
\textit{Bachelor of Engineering in Computer Engineering (CGPA: 9.49/10)} \hfill \textit{2019 - 2023}

%%%%%%% --------------------------------------------------------------------------------------
%%%%%%%  ACADEMIC PROJECTS
%%%%%%% --------------------------------------------------------------------------------------
\vspace{-10pt}
\section*{Projects} 

\subsection*{Etherdocs | Final year graduation project \hfill {\href{https://github.com/DevelopersLeague/EtherDocs/blob/main/Readme.md}{Documentation}} | {\href{https://youtu.be/B_44aJ9hh6U?si=dGItpsQ3AEo_eIxk}{Demo video}}}
\subtext{{\normalsize\normalfont Academic Document Verification Decentralised Application}}
\begin{zitemize}
    \item Developed a blockchain-based decentralized application (DApp) for secure academic document verification, leveraging React.js, Ethereum, smart contracts in Solidity, and IPFS for document storage, achieving 100\% immutability and tamper-proof records.
\end{zitemize}


%%%%%%% --------------------------------------------------------------------------------------
%%%%%%%  SKILLS
%%%%%%% --------------------------------------------------------------------------------------
\section{Skills}

\hskills{Programming Languages, frameworks \& libraries}: Java, Spring Boot, JUnit, Mockito, Cucumber, ReactJS, SQL, Hibernate

\hskills{AWS tools and services}: AWS SDK, Lambda, Kinesis Data Streams (alternative to Apache Kafka), DynamoDB (NoSQL Database), Simple Queue Service (SQS), IAM Roles and Policies, Glue Schema, S3, CloudWatch metrics and alarms, Systems Manager, ECS, Service Catalog, CloudFormation Templates (alternative to Terraform), OpsCenter

\hskills{Key modules}: Java Design Patterns, Microservices, Data Structures, Rest APIs, Object-Oriented Design, Event Driven Design

\hskills{Other}: Git, Maven, Jenkins, Jira, Bitbucket, Gitlab     
% \textcolor{WHITE}{System design, problem-solving, analytical skill} \\
% \end{tabular}
% \vspace{0.1cm}

%%%%%%% --------------------------------------------------------------------------------------
%%%%%%%  AWARDS & HONORS
%%%%%%% --------------------------------------------------------------------------------------
\vspace{-10pt}
\section{Accomplishments}
\begin{zitemize}
    \item \href{https://www.credly.com/badges/c7ea1aea-cdc3-41c5-8fa0-f3e17866f8ff/public_url}{AWS Certified Cloud Practitioner}, Amazon Web Services (AWS) (Jan 2025)
    \item Recognized with Monthly Recognition Award in March 2024 for "Focus on business \& project where the project can excel."
    \item Ranked 2/1700+ on Geeks on GeeksForGeeks platform 
    \href{https://auth.geeksforgeeks.org/college/thadomal-shahani-engineering-college-tsec-mumbai/}{({view standings}).}
    \item Ranked 1st in Project Expo Prakalp-2023 organised by Fr CRCE E-cell.
    \item Stood 3rd in Project Expo organised by TSEC E-cell-2022.
    \end{zitemize}

%%%%%%% ---------------------------- END DOC HERE ---------------------------- %%%%%%% 

\end{document}
